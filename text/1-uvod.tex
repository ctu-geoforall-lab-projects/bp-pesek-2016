\chapter{Úvod}
\label{1-uvod}

Scintilační spektrometry měří radioaktivitu, ale nikoli zeměpisné souřadnice, které těmto hodnotám
náleží. To zajišťují další přístroje. Tyto přístroje však většinou zapisují souřadnice
v~momentu, kdy začíná měření dalších dat, nebo v~momentu jeho skončení. Pro zpracování takových dat
by tedy bylo záhodno mít možnost využití zásuvného modulu, který by tyto posuny alespoň částečně korigoval. 

Zájem o~vývoj takového zásuvného modulu pochází od Státního ústavu radiační ochrany (\zk{SÚRO}).
\zk{SÚRO} se zabývá mimo jiné vyhledáváním lokalit se zvýšenou koncentrací radonu a~vedením centrální
databáze takových míst, poskytováním konzultací, prováděním laboratorních expertíz a~hodnocením
radiační ochrany v~oblasti lékařského ozáření. 

Ústav se rovněž angažuje v~oblasti výzkumných projektů, například „výzkum pokročilých metod detekce,
stanovení a~následného zvládnutí radioaktivní kontaminace s~cílem modernizovat odpovídající části
systému zajištění ochrany obyvatel a~vybraných kritických infrastruktur ČR v~souvislosti
s~radiologickým útokem nebo velkou radiologickou havárií“ nebo „testování nových systémů
hromadného měření radiojódu ve štítné žláze po havárii jaderně energetického zařízení“. \cite{surovyzkum}

Velkou část činností \zk{SÚRO} tvoří též terénní měření a~následné zpracování
těchto dat v~prostředí QGIS, jehož přesnost by měl zlepšit zásuvný model, který bude výstupem této práce. 


