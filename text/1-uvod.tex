\chapter{Úvod}
\label{1-uvod}

Scintilační spektrometry měří radioaktivitu, ale nikoli zeměpisné souřadnice, které těmto hodnotám
náleží. To zajišťují další přístroje. Tyto přístroje však většinou zapisují souřadnice
v~momentu, kdy začíná měření dalších dat, nebo v~momentu jeho skončení. Pro zpracování takových dat
%%% ML: v teto casti bych nepouzival termin zasuvny modul, ale pouze softwarovy nastroj
by tedy bylo záhodno mít možnost využití softwarového nástroje, který by tyto posuny alespoň
částečně korigoval. 

%%% ML: viz poznamka vyse
Zájem o~vývoj takového nástroje pochází od Státního ústavu radiační ochrany (\zk{SÚRO}).
\zk{SÚRO} se zabývá mimo jiné vyhledáváním lokalit se zvýšenou koncentrací radonu a~vedením centrální
databáze takových míst, poskytováním konzultací, prováděním laboratorních expertíz a~hodnocením
radiační ochrany v~oblasti lékařského ozáření. 

Ústav se rovněž angažuje v~oblasti výzkumných projektů, například „výzkum pokročilých metod detekce,
stanovení a~následného zvládnutí radioaktivní kontami\-nace s~cílem modernizovat odpovídající části
systému zajištění ochrany obyvatel a~vybraných kritických infrastruktur ČR v~souvislosti
s~radiologickým útokem nebo velkou radiologickou havárií“ nebo „testování nových systémů
hromadného měření radiojódu ve štítné žláze po havárii jaderně energetického zařízení“. \cite{surovyzkum}

%%% ML: tady bych poprve mluvil o zasuvnem modulu (ten pojem by chtelo vysvetlit)
%%% ML: posledni cast vety zni divne "jehož přesnost", navrhuji preformulovat
%%% ML: odstavec rozsitit, je klicovy pro motivaci (1) nastroj je
%%% treba (2) suro pouziva QGIS (3) vysledek uvahy - navrhnout
%%% rozsirujici nastroj do QGIS pro tuto ulohu (tj. zasuvny modul)
Velkou část činností \zk{SÚRO} tvoří též terénní měření a~následné zpracování
těchto dat v~prostředí QGIS. Zpracovávaná data jsou však zatížena zmiňovaným zpožděním zápisu. 

Optimální by se zdála modifikace zařízení zapisujícího souřadnice. Takovému kroku
ale brání nejen nemožnost dostat se k~podobným právům u~výrobce, nýbrž také empirické zkušenosti
\zk{SÚRO}. Zpracovatelé Ústavu došli ke zjištění, že mnohdy nestačí užít posun, který by se na první
pohled zdál ideální (posouvat body o~polo\-vinu dráhy k~dalšímu bodu); zápis je totiž zatížen dalšími
chybami. Zpracovatel pak potřebuje vyzkoušet několik posunů a~dle rozložení hodnot usoudit, který
z~nich vyhovuje skutečnosti nejvíce. Měřená hodnota by neměla mít skokovitý charakter, ale měla by
v~okolí kritických hodnot nabývat přibližně lineárního rozložení. 

Vzhledem k~tomu, že \zk{SÚRO} zpracovává data v~prostředí QGIS, nabízí se vývoj rozšiřujícího
nástroje právě pro toto prostředí. Zde by měl být implementován jako takzvaný zásuvný modul. S~tím
by souvisela též veřejná dostupnost zásuvného modulu - v~případě potřeby by jej posléze mohla využívat
například i~Armáda České republiky, sahající při některých úkonech též po softwaru QGIS. 

Pro pokročilejší verzi zásuvného modulu by byla vhodná implementace několika variant posunu
a~možností stylovat vstupující data a~porovnávat posunutá data s~daty původními. 


