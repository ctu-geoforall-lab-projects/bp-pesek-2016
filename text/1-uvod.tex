\chapter{Úvod}
\label{1-uvod}

*********************************************************************
CHYBÍ PRVNÍ ČÁST - NEVÍM, ZDA POUŽÍVAT SLOVO LEVELING + DOPLNIT ZDROJE A CITACE
*********************************************************************

Zájem o vývoj takového zásuvného modulu pochází od Státního ústavu radiační ochrany (SÚRO). SÚRO se zabývá mimo jiné vyhledáváním lokalit se zvýšenou koncentrací radonu a vedením centrální databáze takových míst, poskytováním konzultací, prováděním laboratorních expertíz a hodnocením radiační ochrany v oblasti lékařského ozáření. 

Ústav se rovněž angažuje v oblasti výzkumných projektů, na příklad "výzkum pokročilých metod detekce, stanovení a následného zvládnutí radioaktivní kontaminace s cílem modernizovat odpovídající části systému zajištění ochrany obyvatel a vybraných kritických infrastruktur ČR v souvislosti s radiologickým útokem nebo velkou radiologickou havárií" nebo "testování nových systémů hromadného měření radiojódu ve štítné žláze po havárii jaderně energetického zařízení". 

Velkou část činností Ústavu tvoří též terénní měření a následné zpracování v prostředí QGIS. Tato měřená data je třeba dále zpracovávat.


