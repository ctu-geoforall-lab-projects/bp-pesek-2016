\chapter{Použité technologie}
\label{3-technologie}

Třetí kapitola se zabývá užitými technologiemi, stručně představuje jejich historii a~zaměření.
Podrobnější informace o~jednotlivých projektech lze nalézt na oficiálních internetových stránkách. 


\section{Python}
\label{python}

  \begin{figure}[H]
   \centering
	\includegraphics[scale=0.5]{./pictures/python-logo-master-v3-TM.png}
	\caption[Python logo]{Python logo 
      (zdroj: \url{https://www.python.org/static/community_logos/python-logo-master-v3-TM.png})}
      \label{fig:python}
  \end{figure}

Python je open-source multiplatformní programovací interpretovaný jazyk pojme\-novaný podle
%%% ML: "se vyvíjejí verze" zni divne, po tretim precteni jsem se jiz z touto vetou smiril ;-)
kultovního britského komediálního seriálu Monty Pythonův létající cirkus. Momentálně se
vyvíjejí verze třetí řady (3.X). 

První verzi, vydanou roku 1991, navrhl nizozemský programátor a~inženýr Guido van~Rossum za užití kódu
mnoha dalších programátorů. Silnou inspiraci van~Rossum našel v~jazyku ABC. Odtud také plyne
přívětivost jazyka, pro niž je často vyučován a~doporučován jako ideální pro rychlý
(a~přesto kvalitní) vývoj aplikací. 

Zásadní byl pro vývoj rok~2001, kdy vznikla nezisková organizace Python Software
Foundation (\zk{PSF}). \zk{PSF} zaštiťuje další vývoj Pythonu, vlastní jeho
\textit{intelektuální jmění} a~pořádá a~podporuje konference ohledně Pythonu. Sám van~Rossum bývá
pro svůj dohled a~svou neomezenou možnost zasahovat do vývoje nazýván \textit{benevolentní doživotní
diktátor}, španělská inkvizice jazyka Python. 

Python se na první pohled vyznačuje odsazováním kódu. Toho se ve většině jazyků používá jen jako
pomůcky pro přehlednost, v~Pythonu je však odsazování povinné. 

Mezi další specifikace jazyka patří například: 
\begin{itemize}

%%% ML: objektem je vse od trid az po datove typy, funkce je ulozena jako objekt v pameti pocitace i po volani 
	\item Chování funkce: Ta se do svého zavolání uchovává jako objekt. 
	
	\item Proměnné: Není třeba deklarovat jejich typ. To může ušetřit značné množství práce
	v~psaní kódu, ale je třeba si na tuto vlastnost dát pozor (celé číslo dělené celým číslem je
	vždy celé číslo, i kdyby správným výsledkem mělo být číslo desetinné). 
	
	\item Proměnné uvnitř objektu: Není třeba je udávat při vytváření objektu. Lze je založit později. 
%%% ML: provádíme
	\item Dokončení zápisu: V~interaktivním režimu provádím dokončení zápisu prázdným řádkem. 
%%% ML: kompilace, pyc/pyd, bytecode je platformně závislý
	\item Kompilace: Python si automaticky kompiluje moduly do souborů \textit{.pyc}. Zkompilovaný
	modul je platformně nezávislý. 

\end{itemize}


\section{QGIS}
\label{qgis}

  \begin{figure}[H]
    \centering
      \includegraphics[width=120pt]{./pictures/qgis.png}
      \caption[QGIS logo]{QGIS logo 
      (zdroj: \href{https://commons.wikimedia.org/wiki/File:QGIS\_logo.svg}{Wikimedia Commons})}
      \label{fig:qgis}
  \end{figure}

%%% ML: komercni vs. proprietarni
QGIS (zkratka dříve užívaného názvu Quantum~GIS) představuje na poli
geogra\-fic\-kých informačních systémů (\zk{GIS}) open-source
alternativu ke komerčním geografic\-kým informačním systémům
typu~ArcGIS.

Na začátku 21.~století už byly geografické informační systémy běžnou
praxí. Zrodila se tedy potřeba volby svobodného multiplatformního
systému s~širokou podporou formátů geodat. Roku 2002 začal Quantum~GIS
vyvíjet Gary~Sherman, k~němuž se později připojovalo čím dál více
dobrovolníků. Časem projekt zaštítilo též Open Source Geospatial
Foundation (\zk{OSGeo})~–~organizace pro podporu a~vývoj otevřených
geoinformačních technologií a dat vzniklá roku~2006. Verze~1.0 byla
uveřejněna v~lednu~2009.

První verze systému QGIS byly pojmenovávány podle psů, později se
přešlo na jména měsíců Jupiteru a~Saturnu. Momentálně nesou nejnovější
verze názvy měst.  QGIS se ve svém zaměření příliš neliší od běžných
geografických informačních systémů. Uživatel v~něm má možnost
prohlížení, zpracování, tvorby a~editace geodat, nad nimiž může
provádět například SQL~dotazy. Data v~prostředí QGIS mohou být jak
rastrová, tak také vektorová. Co se funkcionality týče, běžnému
%%% ML: komercne rozsirene... 
uživateli dostačuje, ale možností komerčních gigantů typu ArcGIS
nedosahuje.

Síla QGIS vedle volnosti užití tkví především ve značném množství
veřejně přístupných zásuvných modulů~–~tzv.~pluginů. Ty byly zprvu
psány především v~jazyku C++ (stejně jako základní tělo programu),
nyní se čím dál častěji přechází k~jazyku Python, přičemž především
pro grafické uživatelské rozhraní se využívá Qt~knihoven.



\section{Qt Project}
\label{qt}

  \begin{figure}[H]
    \centering
      \includegraphics[width=120pt]{./pictures/qt.png}
      \caption[Qt logo]{Qt logo 
      (zdroj: \url{http://d3hp9ud7yvwzy0.cloudfront.net/wp-content/uploads/2015/02/Qt-logo-medium.png})}
      \label{fig:qt}
  \end{figure}

Qt představuje uživatelsky přívětivé multiplatformní vývojové prostředí s~důrazem na grafické
uživatelské rozhraní (\zk{GUI}). 

Vývoj Qt započali na úsvitu devadesátých let dva programátoři - Haavard Nord a~Eirik
Chambe-Eng. Původně dali své společnosti jméno Quasar technologies, ale brzy společnost
přejmenovali na Trolltech. Pod touto hlavičkou působili až do roku~2008, kdy firmu odkoupila
Nokia. V letech 2011 a~2012 pak probíhal další přesun, tentokráte pod podnik Digia. Digia
nadále spolupracuje s~Qt~Company. 

Qt není programovacím jazykem. Qt je framework psaný v~C++, umožňující ale práci i~v~dalších jazycích
včetně jazyka python (PyQt). Qt klade zesílený důraz na \zk{GUI} a~práci s~ním. Mezi jednotlivými
objekty se dá komunikovat a~vykonávat nad nimi operace pomocí tzv.~signálů a~slotů. \zk{GUI} se ukládá
do souboru s~příponou \textit{.ui}. Pro náš zásuvný modul je důležitá právě tato odnož Qt poskytovaná
samostatně jako Qt~Designer. 




