\chapter{Použité technologie}
\label{3-technologie}




\section{Python}
\label{python}

 


\section{QGIS}
\label{qgis}

QGIS (zkratka dříve užívaného názvu Quantum GIS) představuje na poli geografických informačních systémů (GIS) open-source alternativu ke komerčním geografickým informačním systémům typu ArcGIS. 

Na začátku 21. století už byly geografické informační systémy běžnou praxí. Zrodila se tedy potřeba volby svobodného multiplatformního systému s širokou podporou formátů geodat. Roku 2002 začal Quantum GIS vyvíjet Gary Sherman, k němuž se později připojovalo čím dál více dobrovolníků. Časem projekt zaštítilo též roku 2006 vzniklé Open Source Geospatial Foundation (OSGeo) – organizace pro podporu a vývoj otevřených geoinformačních technologií a dat. Verze 1.0 byla uveřejněna v lednu 2009. 
První verze systému QGIS byly pojmenovávány podle psů, později se přešlo na jména měsíců Jupiteru a Saturnu. Momentálně nesou nejnovější verze názvy měst. 

QGIS se ve svém zaměření příliš neliší od běžných GIS systémů. Uživatel v něm má možnost prohlížení, zpracování, tvorby a editace geodat, nad nimiž může provádět na příklad SQL dotazy. Data v prostředí QGIS mohou být jak rastrová, tak také vektorová. 

Síla QGIS vedle volnosti užití tkví především ve značném množství veřejně přístupných zásuvných modulů – tzv. pluginů. Ty byly zprvu psány především v jazyku C++ (stejně jako základní tělo programu), nyní se čím dál více přechází k jazyku Python, přičemž především pro grafické uživatelské rozhraní se využívá Qt knihoven. 



\section{Qt Project}
\label{qt}