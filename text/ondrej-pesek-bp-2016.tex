%%%%%%%%%%%%%%%%%%%%%%%%%%%%%%%%%%%%%%%%%%%%%%%%%%%%%%%%%%%%%%%%%%%%%%%%%%%%%%%%%%%%%%%%%%%%%%%%%%%%%%%
%%													%%
%% 	BAKALÁŘSKÁ PRÁCE - Posun letecky měřených bodů po trajektorii v prostředí~QGIS 				%%
%% 				 Ondřej Pešek							%%
%%													%%
%% pro formátování využita šablona: http://geo3.fsv.cvut.cz/kurzy/mod/resource/view.php?id=775 	%%
%%													%%
%%%%%%%%%%%%%%%%%%%%%%%%%%%%%%%%%%%%%%%%%%%%%%%%%%%%%%%%%%%%%%%%%%%%%%%%%%%%%%%%%%%%%%%%%%%%%%%%%%%%%%% 

\documentclass[%
  12pt,         			% Velikost základního písma je 12 bodů
  a4paper,      			% Formát papíru je A4
  twoside,       			% Oboustranný tisk
  pdftex,				    % překlad bude proveden programem 'pdftex' do PDF
%%%  draft
]{report}       			% Dokument třídy 'zpráva'
%

\usepackage[czech, english]{babel}	% použití češtiny, angličtiny
\usepackage[utf8]{inputenc}		% Kódování zdrojových souborů je UTF8

\usepackage[square,sort,comma,numbers]{natbib}

\usepackage{caption}
\usepackage{subcaption}
\captionsetup{font=small}
\usepackage{enumitem} 
\setlist{leftmargin=*} % bez odsazení

\makeatletter
\setlength{\@fptop}{0pt}
\setlength{\@fpbot}{0pt plus 1fil}
\makeatletter

\usepackage[dvips]{graphicx}   
\usepackage{color}
\usepackage{transparent}
\usepackage{wrapfig}
\usepackage{float} 

\usepackage{cmap}           
\usepackage[T1]{fontenc}    

\usepackage{textcomp}
\usepackage[compact]{titlesec}
\usepackage{amsmath}
\addtolength{\jot}{1em} 

\usepackage{chngcntr}
\counterwithout{footnote}{chapter}

\usepackage{acronym}

\usepackage[
    unicode,                
    breaklinks=true,        
    hypertexnames=false,
    colorlinks=true, % true for print version
    citecolor=black,
    filecolor=black,
    linkcolor=black,
    urlcolor=black
]{hyperref}         

\usepackage{url}
\usepackage{fancyhdr}

\usepackage[
  cvutstyle,          
  bachelor           
]{thesiscvut}


\newif\ifweb
\ifx\ifHtml\undefined % Mimo HTML.
    \webfalse
\else % V HTML.
    \webtrue
\fi 

\renewcommand{\figurename}{Obrázek}
\def\figurename{Obrázek}

%%%%%%%%%%%%%%%%%%%%%%%%%%%%%%%%%%%%%%%%%%%%%%%%%%%%%%%%%%%%%%%%%
%%%%%%%%%%% Definice informací o dokumentu  %%%%%%%%%%%%%%%%%%%%%
%%%%%%%%%%%%%%%%%%%%%%%%%%%%%%%%%%%%%%%%%%%%%%%%%%%%%%%%%%%%%%%%%

%% Název práce
\nazev{Posun letecky měřených bodů po trajektorii v~prostředí~QGIS}
{Aerial data move on a~trajectory in QGIS software}

%% Jméno a příjmení autora
\autor{Ondřej}{Pešek}

%% Jméno a příjmení vedoucího práce včetně titulů
\garant{Ing.~Martin~Landa,~Ph.D.}

%% Označení oboru studia
\oborstudia{Geodézie, kartografie a~geoinformatika}{}

%% Označení ústavu
\ustav{Katedra geomatiky}{}

%% Rok obhajoby
\rok{2016}

%Mesic obhajoby
\mesic{červen}

%% Místo obhajoby
\misto{Praha}

%% Abstrakt
%%% ML: rozsirit, kopie zadani je pro abstrakt malo
\abstrakt 
{Cílem bakalářské práce je návrh softwarového nástroje umožňujícího posun letecky měřených bodů po
trajektorii. Takový nástroj je třeba z~toho důvodu, že přístroj při
leteckých měřeních zapisuje souřadnice s~určitým zpožděním. V~praktické části práce se objevuje jeho
implementace jako tzv.~zásuvného modulu do prostředí open source projektu QGIS s~využitím grafického
frameworku Qt. }%
{The object of bachelor thesis is creation of software tool for moving aerial data by their trajectory
(GPS position lag correction). The reason for this tool is that the instrument for
aerial data surveying is recording data with some delay. In practical part of bachelor thesis is
implementation of this tool as plugin into open source project QGIS using graphical framework Qt. 
 }

%% Klíčová slova
\klicovaslova
{GIS, QGIS, zásuvný~modul, python, letecká~data}%
{GIS, QGIS, plugin, python, aerial~data}

%%%%%%%%%%%%%%%%%%%%%%%%%%%%%%%%%%%%%%%%%%%%%%%%%%%%%%%%%%%%%%%%%%%%%%%%

%%%%%%%%%%%%%%%%%%%%%%%%%%%%%%%%%%%%%%%%%%%%%%%%%%%%%%%%%%%%%%%%%%%%%%%%
%% Nastavení polí ve Vlastnostech dokumentu PDF
%%%%%%%%%%%%%%%%%%%%%%%%%%%%%%%%%%%%%%%%%%%%%%%%%%%%%%%%%%%%%%%%%%%%%%%%
\nastavenipdf
%%%%%%%%%%%%%%%%%%%%%%%%%%%%%%%%%%%%%%%%%%%%%%%%%%%%%%%%%%%%%%%%%%%%%%%

%%% Začátek dokumentu
\begin{document}

\catcode`\-=12  % pro vypnuti aktivniho znaku '-' pouzivaneho napr. v \cline 

% aktivace záhlaví
\zahlavi

% předefinování vzhledu záhlaví
\renewcommand{\chaptermark}[1]{%
	\markboth{\MakeUppercase
	{%
	\thechapter.%
	\ #1}}{}}

% Vysázení přebalu práce
%\vytvorobalku

% Vysázení titulní stránky práce
\vytvortitulku

% Vysázení listu zadani
\stranka{}%
	{\sffamily\Huge\centering\ ZDE VLOŽIT LIST ZADÁNÍ}%
	{\sffamily\centering Z~důvodu správného číslování stránek}

% Vysázení stránky s abstraktem
\vytvorabstrakt

% Vysázení prohlaseni o samostatnosti
\vytvorprohlaseni

% Vysázení poděkování
\stranka{%nahore
       }{%uprostred
       }{%dole
       \sffamily
	\begin{flushleft}
		\large
		\MakeUppercase{Poděkování}
	\end{flushleft}
	\vspace{1em}
		%\noindent
	\par\hspace{2ex}
	{Chtěl bych poděkovat vedoucímu práce, Ing.~Martinu Landovi,~PhD., za připomínky a~pomoc při zpracování této práce. Dále bych chtěl poděkovat Martinu~Joskovi za to, že mě poprosil, zda bych mu nepoděkoval. }
}

% Vysázení obsahu
\obsah

% Vysázení seznamu obrázků
\seznamobrazku

% Vysázení seznamu tabulek
%\seznamtabulek

% jednotlivé kapitoly
\chapter{Úvod}
\label{1-uvod}

Scintilační spektrometry měří radioaktivitu, ale nikoli zeměpisné souřadnice, které těmto hodnotám
náleží. To zajišťují další přístroje. Tyto přístroje však většinou zapisují souřadnice
v~momentu, kdy začíná měření dalších dat, nebo v~momentu jeho skončení. Pro zpracování takových dat
%%% ML: v teto casti bych nepouzival termin zasuvny modul, ale pouze softwarovy nastroj
by tedy bylo záhodno mít možnost využití softwarového nástroje, který by tyto posuny alespoň
částečně korigoval. 

%%% ML: viz poznamka vyse
Zájem o~vývoj takového nástroje pochází od Státního ústavu radiační ochrany (\zk{SÚRO}).
\zk{SÚRO} se zabývá mimo jiné vyhledáváním lokalit se zvýšenou koncentrací radonu a~vedením centrální
databáze takových míst, poskytováním konzultací, prováděním laboratorních expertíz a~hodnocením
radiační ochrany v~oblasti lékařského ozáření. 

Ústav se rovněž angažuje v~oblasti výzkumných projektů, například „výzkum pokročilých metod detekce,
stanovení a~následného zvládnutí radioaktivní kontami\-nace s~cílem modernizovat odpovídající části
systému zajištění ochrany obyvatel a~vybraných kritických infrastruktur ČR v~souvislosti
s~radiologickým útokem nebo velkou radiologickou havárií“ nebo „testování nových systémů
hromadného měření radiojódu ve štítné žláze po havárii jaderně energetického zařízení“. \cite{surovyzkum}

%%% ML: tady bych poprve mluvil o zasuvnem modulu (ten pojem by chtelo vysvetlit)
%%% ML: posledni cast vety zni divne "jehož přesnost", navrhuji preformulovat
%%% ML: odstavec rozsitit, je klicovy pro motivaci (1) nastroj je
%%% treba (2) suro pouziva QGIS (3) vysledek uvahy - navrhnout
%%% rozsirujici nastroj do QGIS pro tuto ulohu (tj. zasuvny modul)
Velkou část činností \zk{SÚRO} tvoří též terénní měření a~následné zpracování
těchto dat v~prostředí QGIS. Zpracovávaná data jsou však zatížena zmiňovaným zpožděním zápisu. 

Optimální by se zdála modifikace zařízení zapisujícího souřadnice. Takovému kroku
ale brání nejen nemožnost dostat se k~podobným právům u~výrobce, nýbrž také empirické zkušenosti
\zk{SÚRO}. Zpracovatelé Ústavu došli ke zjištění, že mnohdy nestačí užít posun, který by se na první
pohled zdál ideální (posouvat body o~polo\-vinu dráhy k~dalšímu bodu); zápis je totiž zatížen dalšími
chybami. Zpracovatel pak potřebuje vyzkoušet několik posunů a~dle rozložení hodnot usoudit, který
z~nich vyhovuje skutečnosti nejvíce. Měřená hodnota by neměla mít skokovitý charakter, ale měla by
v~okolí kritických hodnot nabývat přibližně lineárního rozložení. 

Vzhledem k~tomu, že \zk{SÚRO} zpracovává data v~prostředí QGIS, nabízí se vývoj rozšiřujícího
nástroje právě pro toto prostředí. Zde by měl být implementován jako takzvaný zásuvný modul. S~tím
by souvisela též veřejná dostupnost zásuvného modulu - v~případě potřeby by jej posléze mohla využívat
například i~Armáda České republiky, sahající při některých úkonech též po softwaru QGIS. 

Pro pokročilejší verzi zásuvného modulu by byla vhodná implementace několika variant posunu
a~možností stylovat vstupující data a~porovnávat posunutá data s~daty původními. 



\chapter{Teoretický základ}
\label{2-teorie}



\section{Scintilační spektrometrie}
\label{spektrometrie}



\section{Sběr souřadnic a potřeba posunu}
\label{potreba posunu}




\section{Posun}
\label{posun}

\subsection{O hodnotu}
\label{by_points}


\subsection{O konstantní vzdálenost}
\label{by_distance}


\subsection{O konstantní čas (o proměnnou vzdálenost)}
\label{by_seconds}


\section{Problémy}
\label{problemy}

\subsection{Elipsoid}
\label{elipsoid}


\subsection{První geodetická úloha}
\label{prvnigu}



\chapter{Použité technologie}
\label{3-technologie}




\section{Python}
\label{python}

 


\section{QGIS}
\label{qgis}

QGIS (zkratka dříve užívaného názvu Quantum GIS) představuje na poli geografických informačních systémů (GIS) open-source alternativu ke komerčním geografickým informačním systémům typu ArcGIS. 

Na začátku 21. století už byly geografické informační systémy běžnou praxí. Zrodila se tedy potřeba volby svobodného multiplatformního systému s širokou podporou formátů geodat. Roku 2002 začal Quantum GIS vyvíjet Gary Sherman, k němuž se později připojovalo čím dál více dobrovolníků. Časem projekt zaštítilo též roku 2006 vzniklé Open Source Geospatial Foundation (OSGeo) – organizace pro podporu a vývoj otevřených geoinformačních technologií a dat. Verze 1.0 byla uveřejněna v lednu 2009. 
První verze systému QGIS byly pojmenovávány podle psů, později se přešlo na jména měsíců Jupiteru a Saturnu. Momentálně nesou nejnovější verze názvy měst. 

QGIS se ve svém zaměření příliš neliší od běžných GIS systémů. Uživatel v něm má možnost prohlížení, zpracování, tvorby a editace geodat, nad nimiž může provádět na příklad SQL dotazy. Data v prostředí QGIS mohou být jak rastrová, tak také vektorová. 

Síla QGIS vedle volnosti užití tkví především ve značném množství veřejně přístupných zásuvných modulů – tzv. pluginů. Ty byly zprvu psány především v jazyku C++ (stejně jako základní tělo programu), nyní se čím dál více přechází k jazyku Python, přičemž především pro grafické uživatelské rozhraní se využívá Qt knihoven. 



\section{Qt Project}
\label{qt}
\chapter{Zásuvný modul}
\label{4-plugin}

Čtvrtá kapitola popisuje vývoj zásuvného modulu a~rozebírá důležité části kódu. 


\section{Obsah CSV}
\label{obsah}

\section{Tělo zásuvného modulu}
\label{telo}

\section{Posun o hodnoty}
\label{by_points}

\section{Posun o konstantní vzdálenost}
\label{by_distance}

\subsection{Výpočet azimutu}
\label{azimut}

\subsection{První geodetická úloha}
\label{prvniguplugin}

\section{Posun o konstantní čas (proměnnou vzdálenost)}
\label{by_seconds}

\section{Licence}
\label{licence}




\chapter{Závěr}
\label{zaver}

Cílem této bakalářské práce byla tvorba
softwarového nástroje umožňujícího posun letecky
měřených bodů po trajektorii a~jeho implementace
do prostředí QGIS jako tzv.~zásuvného modulu. 

Celý projekt se zakládal na pracích Státního
ústavu radiační ochrany (\zk{SÚRO}). Ten dosud prováděl posun
tak, že každému bodu pouze přiřadil souřadnice bodu
předcházejícího či následujícího (s~libovolným
krokem). Přihlédneme-li k~bouchoři, kterým podobný posun
o~hodnoty jest, můžeme říci, že bylo zadání
nejen splněno, ale na základě návrhů \zk{SÚRO}
dovedeno ještě dále. 

V~současnosti zásuvný modul umožňuje posun dat
nikoli jen o~hodnoty, nýbrž také o~konstantní
vzdálenost či o~čas (tj.~o~vzdálenost
s~uvážením rychlosti). V~grafic\-kém
rozhraní si může uživatel též zvolit styl,
který by chtěl pro zobrazení vstupních
a~výstupních dat použít. 

V~čase budoucím se nabízejí některá rozšíření
stávající funkcionality. Počítá se s~implementací
více stylů běžně užívaných při zpracování. Bylo by
rovněž vhodné dát uživateli možnost zvolit
referenční elipsoid ad libitum; momentálně
probíhají veš\-keré výpočty na elipsoidu \zk{WGS84}.
Veškerá dostupná data sice obsahují zeměpisné souřadnice
právě na tomto elipsoidu, nelze ale vyloučit, že
se v~cizině vyskytují i~nějaké adventivní případy.

%%% ML: zde by se hodil odkaz na xml soubor QGIS repozitare
Zásuvný modul byl, prozatím pouze pod hlavičkou \textit{experimentální},
uveřejněn v~repositáři OSGeoREL ({\tt http://geo.fsv.cvut.cz/osgeorel/qgis-plugins.xml}). V~\zk{SÚRO} probíhá jeho testování již od dubna~2016.




% Vysázení seznamu zkratek

\begin{seznamzkratek}{ABCDE}

      \novazkratka{OSGeo}
	      {OSGeo}
	      {Open Source Geospatial Foundation}

      \novazkratka{SQL}	
	      {SQL}
	      {Structured Query Language}

\end{seznamzkratek}

% Literatura
\nocite{*}
\def\refname{Literatura}
\bibliographystyle{mystyle}
\bibliography{literatura}


% Začátek příloh
\prilohy

% Vysázení seznamu příloh
\seznampriloh

% Vložení souboru s přílohami
%%% ML: text by mel projit kompletni revizi 

%%%%%%%%%%%%%%%%%%%%%%%%%%%%%%%%%%%%%%%%%%%%%%%%%%%%%%%%%%%%%%%%%%%%%%%%%%%%%%%%%%%
%%                 PŘÍLOHA - UŽIVATELSKÁ PŘÍRUČKA                                %%
%%%%%%%%%%%%%%%%%%%%%%%%%%%%%%%%%%%%%%%%%%%%%%%%%%%%%%%%%%%%%%%%%%%%%%%%%%%%%%%%%%%
\chapter{User guide}
\label{user-guide}

%%% ML: works with -> usage of the plugin can change accross...
This user guide is written for plugin using in QGIS 2.12. There is a~possibility that works with plugin
will be different in other versions of QGIS. 

\section{Loading of plugin}
\label{plugin-load}

* 
* 

* 
* TUHLE ČÁST JEŠTĚ MUSÍM DOPSAT - NEJDŘÍV JE POTŘEBA PLUGIN ZE SLOŽKY SUROLEVELING FUNKČNĚ PŘESUNOUT DO GPS POSITION LAG CORRECTION, ABYCH VĚDĚL, JAK SE BUDE V NABÍDCE PLUGIN JMENOVAT
*

* 
*

\section{Work with plugin}
\label{work}

%%% ML: preformulovat In this section will be described graphical user
%%% interface and primary functionality of the plugin
In this section will be described primary functions and working of graphical user interface of plugin. 

  \begin{figure}[H]
   \centering
	\includegraphics[scale=0.75]{./pictures/gui.png}
	\caption[GUI]{GUI of the plugin}
      \label{fig:gui}
  \end{figure}

\subsection{Input and output defining}
\label{input-output}

The basic elements are of course input and output. 

In input, you have to define the~file you want to work with. You can write the~path to this file or
there is the~button with {\tt ...}. If you click on this button, you will get the~interface where
you can choose the~path to your file. Click on OK will insert this path to lineedit in basic GUI. Click
on Cancel will interrupt the~interface for choose, and the~content of lineedit will not be changed. 

  \begin{figure}[H]
   \centering
	\includegraphics[scale=0.75]{./pictures/input.png}
	\caption[Loading input]{Loading input}
      \label{fig:input}
  \end{figure}

The~same work is with output, but there you define the~path where the~new file will be created.
You can manually define the~path or use the~browse button. You will again get a~new interface, this
time created for choosing folder. 

  \begin{figure}[H]
   \centering
	\includegraphics[scale=0.75]{./pictures/output.png}
	\caption[Choosing output directory]{Choosing output directory}
      \label{fig:output}
  \end{figure}

\subsection{Styling}
\label{styling}

There is also additional possibility of styling your points for better visualisation. 

In the~default GUI is the~button \textit{No style}. It means that you have not defined any style
and points will be created in default QGIS style. If you want your own style, you have to click on
this button. You will get the~browsing interface. You are automatically directed to folder
\textit{styles} in the~plugin folder; you have to choose qml file and click OK. Click on Cancel will
interrupt the interface and set again \textit{No style}. 

  \begin{figure}[H]
   \centering
	\includegraphics[scale=0.75]{./pictures/style.png}
	\caption[Choose style]{Choose style}
      \label{fig:style}
  \end{figure}

If you have chosen your own style, you would see its name on the~style button. 

  \begin{figure}[H]
   \centering
	\includegraphics[scale=0.75]{./pictures/style-defined.png}
	\caption[Used style]{Used style}
      \label{fig:used-style}
  \end{figure}

In the~default version of plugin there are two default styles. Styling in those styles is based on
column mereni – one style for higher and one style for lower values. 

\subsection{Showing input}
\label{show}

Next to input is the button \textit{Show}. 

Show button is just to visualise the input file as layer in QGIS. It has no effect on the run of plugin
and it is not necessary for the~run, but it allows you to visualise the~input file in the same style as
the~output file. It gives you also the~opportunity to compare the~moved values with the~original ones. 

This button was created on request by users who prefer to visualise both the~input and the~output in
the~same interface. 

If input file does not exists, plugin will raise the~error message. 

%%% ML: Shift
\subsection{Move}
\label{move}

Move will be done by clicking on button \textit{Solve}. Before moving you have to define few more things. 

In combobox you can choose the~units of move – each presents other type of move; values mean move
by values, meters mean move by constant distance and seconds mean move by constant time/variable
distance considering current velocity. 

%%% ML: nepouzivat lineedit v textu
In the~appropriate lineedit you have to insert value of move. For values it should be integer (it
does not make sense to move point by 1.5 values because nothing like 1.5 value does not exist), for
other moves it should be integer or float. 

For wrong input, plugin will raise the~error message. 

\subsection{Dependencies}
\label{dependencies}

To avoid some unnecessary errors, there are some dependencies in the~plugin GUI. 

The~first is \textit{Show} button. This button is
disabled until you have wrote anything into input lineedit. 

  \begin{figure}[H]
   \centering
	\includegraphics[scale=0.75]{./pictures/show.png}
	\caption[Enabled Show button]{Enabled Show button}
      \label{fig:show}
  \end{figure}

The~second is button \textit{Solve}. You can’t solve anything until you have defined input, output and
value of move.

  \begin{figure}[H]
   \centering
	\includegraphics[scale=0.75]{./pictures/solve.png}
	\caption[Enabled Solve button]{Enabled Solve button}
      \label{fig:solve}
  \end{figure}

There is also implemented shortcut for defining output directory. Many users wants to save output file to
the~directory from which was read the~input file, so when you choose input file, the~directory will
be automatically copied into output lineedit, just with added \textit{\_moved} into filename. 


% Konec dokumentu
\end{document}
