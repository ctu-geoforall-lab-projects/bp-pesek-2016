\chapter{Teoretický základ}
\label{2-teorie}

Tato kapitola si klade za cíl seznámit čtenáře s~teoretickými základy měření radioaktivity, z~nichž
plyne potřeba posunu. Dále pak načrtne způsoby, kterými bude možno body posouvat, a~popíše
nejdůležitější problémy, s nimiž se při takovém posunu lze setkat. 

\section{Scintilační spektrometrie}
\label{spektrometrie}

\subsection{Detektorová část}
\label{detektor}

\subsubsection{Scintilační krystal}
\label{krystal}

\subsubsection{Fotonásobič}
\label{fotonasobic}

\subsection{Analyzační část}
\label{analyzator}


\section{Sběr souřadnic a potřeba posunu}
\label{potreba posunu}


\section{Posun}
\label{posun}


\section{Problémy}
\label{problemy}

\subsection{Elipsoid}
\label{elipsoid}


\subsection{První geodetická úloha}
\label{prvnigu}


