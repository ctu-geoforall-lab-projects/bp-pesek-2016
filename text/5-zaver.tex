\chapter{Závěr}
\label{zaver}

Cílem této bakalářské práce byla tvorba
softwarového nástroje umožňujícího posun letecky
měřených bodů po trajektorii a~jeho implementace
do prostředí QGIS jako tzv.~zásuvného modulu. 

Celý projekt se zakládal na pracích Státního
ústavu radiační ochrany (\zk{SÚRO}). Ten dosud prováděl posun
tak, že každému bodu pouze přiřadil souřadnice bodu
předcházejícího či následujícího (s~libovolným
krokem). Přihlédneme-li k~bouchoři, kterým podobný posun
o~hodnoty jest, můžeme říci, že bylo zadání
nejen splněno, ale na základě návrhů \zk{SÚRO}
dovedeno ještě dále. 

V~současnosti zásuvný modul umožňuje posun dat
nikoli jen o~hodnoty, nýbrž také o~konstantní
vzdálenost či o~čas (tj.~o~vzdálenost
s~uvážením rychlosti). V~grafic\-kém
rozhraní si může uživatel též zvolit styl,
který by chtěl pro zobrazení vstupních
a~výstupních dat použít. 

V~čase budoucím se nabízejí některá rozšíření
stávající funkcionality. Počítá se s~implementací
více stylů běžně užívaných při zpracování. Bylo by
rovněž vhodné dát uživateli možnost zvolit
referenční elipsoid ad libitum; momentálně
probíhají veš\-keré výpočty na elipsoidu \zk{WGS84}.
Veškerá dostupná data sice obsahují zeměpisné souřadnice
právě na tomto elipsoidu, nelze ale vyloučit, že
se v~cizině vyskytují i~nějaké adventivní případy.

%%% ML: zde by se hodil odkaz na xml soubor QGIS repozitare
Zásuvný modul byl, zatím pod hlavičkou \textit{experimentální},
uveřejněn v~repositáři OSGeoREL ({\tt http://geo.fsv.cvut.cz/osgeorel/qgis-plugins.xml}). V~\zk{SÚRO} probíhá již od dubna~2016 jeho testování.


