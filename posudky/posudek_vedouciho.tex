\documentclass[czech,11pt,a4paper]{article}
\usepackage[utf8]{inputenc}
\usepackage{a4wide}
\usepackage[pdftex,breaklinks=true,colorlinks=true,urlcolor=blue,
  pagecolor=black,linkcolor=black]{hyperref}
\usepackage[czech]{babel}

\pagestyle{empty}

\begin{document}

\begin{center}
  {\Large --- Posudek vedoucího bakalářské práce ---}
\end{center}

\vspace{.2cm}

\noindent \begin{tabular}{rp{.9\textwidth}}
  {\bf Název:} & Posun letecky měřených bodů po trajektorii v prostředí QGIS \\
  {\bf Student:} & Ondřej Pešek \\
  {\bf Vedoucí:} & Ing. Martin Landa, Ph.D. \\
  {\bf Fakulta:} & Fakulta stavební ČVUT v Praze \\ 
  {\bf Katedra:} & Katedra geomatiky \\
  {\bf Oponent:} & RNDr. Marcel Ohera \\
  {\bf Pracoviště oponenta:} & EnviMO, Vlčnovská 16, 628 00 Brno \\
\end{tabular}

\vspace{1cm}

Zadání práce vzešlo z objednávky od kolegů ze Státního ústavu radiační
ochrany (dále SURO). Jejím cílem bylo navrhnout nástroj umožňující
posun letecky měřených bodů po trajektorii. Odůvodnění potřeby
takového nástroje je vysvětleno v kapitole 2.2. Pro jeho vývoj byla
použita open source platforma QGIS, která je na SURO současně i
certifikována. Tento bod je důležitým předpokladem pro nasazení v
prostředí tohoto ústavu. Výsledkem je nástroj (tzv. zásuvný modul) pro
QGIS volně dostupný pod licencí GNU GPL, který je ze strany SURO dále
testován a plánován k využití do ostrého provozu. Aktuálně je dostupný
z repositáře OSGeoREL (Open Source Geospatial Research and Education
Laboratory) na ČVUT v Praze, v budoucnu se počítá s jeho začleněním do
oficiálního repositáře QGIS. \newline

Student musel nejprve vstřebat nutný teoretický základ potřebný pro
další zpracování praktické části vedoucí k výše zmíněnému zásuvnému
modulu QGIS, a to úvod do scintilační spektrometrie a aparátu
potřebného pro výpočet posunu souřadnic na referenční ploše elipsoidu,
tj. první geodetické úlohy. Použité technologie vycházejí z platformy
QGIS. Pro vývoj nástroje byl použit programovací jazyk Python a
grafický framework PyQt. Výsledný produkt je popsán v kapitole 4 a
příloze A, která je vzhledem k tomu, že bude využita pro základ
dokumentace nástroje, zpracována na rozdíl od zbytku práce v jazyce
anglickém. \newline

Jako vedoucí práce si cením především přístupu studenta. Na pravidelné
konzultace dochá\-zel vždy připraven, stanovené úkoly plnil s lehkostí
a samozřejmostí. Tento přístup bohužel není u našich studentů obvyklý
a bylo by mi potěšením se s ním setkávat častěji než
výjimečně. \newline

Zmínku si zaslouží jazyková stránka práce, kde se projevuje hravost
autora a jeho zájem o český jazyk. Skloubit archaická slova s jazykem
současné technické práce se mu podařilo především v teoretické
části. V praktické části naopak působí tato snaha na několika místech
kostrbatě a jaksi nuceně. Výsledkem je nicméně čtivá práce, která
svého čtenáře kromě obsahu zaujme i svým jazykem. Příkladem může být
věta na předělu stran 12 a 13, viz. poznámka pod čarou číslo 2. Osobně
jsem text práce četl s potěšením a radostí, což nebývá u technických
prací obvyklé. \newline

\newpage

Na základě výše uvedeného doporučuji předloženou práci k obhajobě a
hodnotím ji klasifikačním stupněm

\begin{center}
  {\bf --- A (výborně) --- }
\end{center}

\noindent a navrhuji udělení pochvaly za vynikající zpracování závěrečné práce.

\vskip 2cm

\begin{tabular}{lp{.2\textwidth}r}
& & \ldots\ldots\ldots\ldots\ldots\ldots\ldots \\
V~Praze dne 24. června 2016 & & Ing. Martin Landa, Ph.D. \\
& & Fakulta stavební, ČVUT v Praze \\
\end{tabular}

\end{document}
